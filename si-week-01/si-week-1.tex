%%%%%%%%%%%%%%%%%%%%%%%%%%%%%%%%%%%%%%%%%%%%%%%%
% 1. document class
%%%%%%%%%%%%%%%%%%%%%%%%%%%%%%%%%%%%%%%%%%%%%%%%
 
\documentclass[a4paper,12pt]{article}

%%%%%%%%%%%%%%%%%%%%%%%%%%%%%%%%%%%%%%%%%%%%%%%%
% 2. packages
%%%%%%%%%%%%%%%%%%%%%%%%%%%%%%%%%%%%%%%%%%%%%%%%

\usepackage[top = 2.5cm, bottom = 2.5cm, left = 2.5cm, right = 2.5cm]{geometry} 

\usepackage[T1]{fontenc}
\usepackage[utf8]{inputenc}
\usepackage{amsmath}
\usepackage{cancel}
\usepackage{amssymb}
\usepackage{multirow}
\usepackage{booktabs}
\usepackage{graphicx} 
\usepackage{setspace}
\setlength{\parindent}{0in}
\usepackage{float}
\usepackage{fancyhdr}
\usepackage[colorlinks=true,linkcolor=blue]{hyperref}
\usepackage{amsmath}
\usepackage{tikz}
\usetikzlibrary{tikzmark}

%%%%%%%%%%%%%%%%%%%%%%%%%%%%%%%%%%%%%%%%%%%%%%%%
% 3. header (and footer)
%%%%%%%%%%%%%%%%%%%%%%%%%%%%%%%%%%%%%%%%%%%%%%%%

\pagestyle{fancy}
\fancyhf{}

\lhead{\footnotesize }
\rhead{\footnotesize Collins \thepage}
\cfoot{\footnotesize} 

%%%%%%%%%%%%%%%%%%%%%%%%%%%%%%%%%%%%%%%%%%%%%%%%
% 4. the document
%%%%%%%%%%%%%%%%%%%%%%%%%%%%%%%%%%%%%%%%%%%%%%%%

\begin{document}

%%%%%%%%%%%%%%%%%%%%%%%%%%%%%%%%%%%%%%%%%%%%%%%%
% title section of the document
%%%%%%%%%%%%%%%%%%%%%%%%%%%%%%%%%%%%%%%%%%%%%%%%

\thispagestyle{empty}

\begin{tabular}{p{15.5cm}}
\\ Collin Collins \\
MATH 3400\\
SI Session 1 Practice Problems and Solutions\\
24 January 2023 \\
\hline

\end{tabular} 

\subsection*{Problem 1.}

Find the solution of the following IVP
$$ y' = \frac{x + xy}{(x+1)^2 - 1}\quad;\quad y(0)=1 $$

Try it out before looking at the solution

\pagebreak

\subsection*{Solution to 1.}
Let's profile this differential equation based on the types that we have learned.\\

This cannot be an autonomous differential equation because the derivative of $y$ depends on our independent variable, $x$.\\

Let's check if it is separable. To do this we will first factor out an $x$ from the numerator and attempt to split the right-hand-side into the product of $h(x)$ and $g(y)$:

$$ y' = \frac{x(1+y)}{(x+1)^2 - 1} \quad\rightarrow\quad y' = \underbrace{\left(\frac{x}{(x+1)^2-1}\right)}_{h(x)}\underbrace{\left(1+y\right)}_{g(y)}.
$$
Now, let's use Leibniz notation and put $y$'s and $dy$'s on one side and put $x$'s and $dx$'s on the other side.
$$ \frac{dy}{dx} = \left(\frac{x}{(x+1)^2-1}\right){\left(1+y\right)} \quad\rightarrow\quad \frac{1}{1 + y}dy = \frac{x}{(x+1)^2-1}dx. $$
From here, we can remove the $dy$ and $dx$ by integrating both sides.
$$ \int \frac{1}{1 + y}dy = \int \frac{x}{(x+1)^2-1}dx \quad\rightarrow\quad \ln{|1 + y|} = \int \frac{x}{(x+1)^2-1}dx, $$
Let's take a moment and deal with the integral in $x$ separately. We will perform a $u$-sub, letting $u=x+1$ and $du = 1$.
$$ \int \frac{u - 1}{u^2 - 1}du \quad\rightarrow\quad \int \frac{u-1}{(u+1)(u-1)}du \quad\rightarrow\quad  \int \frac{1}{u+1}du = \ln{|u+1|} + C $$
Let's resubstitute $x+1$ for the $u$:
$$ \ln{|(x + 1) + 1|} + C \quad\rightarrow\quad \text{RHS} = \ln{|x + 2|} + C. $$
We can now set the left-hand-side that involves the $y$: $\ln{|1 + y|}$ equal to the right-hand-side that involves the $x$.
$$ \ln{|1 + y|} = \ln{|x + 2|} + C \quad\overset{\text{exponentiate}}\rightarrow\quad |1 + y| = e^{\ln{|x+2|} + C} \quad\overset{C_2=e^{C}}\rightarrow\quad |1+y| = C_2|x+2|.$$
Let's cancel the absolute value signs from both sides:
$$ 1 + y = C_2(x + 2) \quad\rightarrow\quad y_g = C_2(x + 2) - 1. $$
Notice that the $y_g$ indicates that this is the general solution. We're given an initial condition, however. $y(0)=1$ is our initial condition. The easiest way that I've heard of applying this initial condition to find our constant of integration (and thus our specific solution) is to treat it as an ordered pair: $(x=0, y=1)$. Anywhere we see and $x$, we put a 0. Anywhere we see a $y$ we put a 1.\\

$$[1] = C_2([0] + 2) - 1 \quad\rightarrow\quad 1 = 2C_2 - 1 \quad\therefore\quad C_2 = 1,$$
$$ y_s(x) = (x + 2)-1 \quad\rightarrow\quad \boxed{y_s(x) = x + 1} $$
\\
--------------------------------------------------------------------------------------------------------------------

\pagebreak

\subsection*{2.} 
Sketch a phase diagram with arrows on the number line provided and answer the questions regarding the following differential equation:
$$ y^{\prime} = 6(y-1)(y^2+3y+2). $$
(a) What type of differential equation is this and why?\\
(b) Find the critical points and describe the equilibrium solutions.\\
(c) Draw a phase line for the differential equation.\\
(d) What is the stability of each solution?\\
(e) Determine the limits below, for a solution with initial condition $y(0)=-\frac{3}{4}$
$$
\begin{aligned}
& \lim _{t \rightarrow \infty} y(t)=  \quad\quad\quad\quad \lim _{t \rightarrow-\infty} y(t)=
\end{aligned}
$$\\
\\
\\
\\
\\
\\

$$ ----------------------------- $$
\pagebreak

\subsubsection*{Solution to 2:}
\textbf{(a)} The differential equation is an autonomous equation because it lacks dependance on the independent variable, $t$.\\

\textbf{(b)} The equilibrium solutions occur when $y^{\prime}$ = 0, which occurs when any of the factors of this differential equation are equal to zero. Let's factor the quadratic on the right:
$$ y^{\prime} = 6(y-1)\underbrace{(y^2+3y+2)}_{(y + 2)(y + 1)}, $$
\text{By the zero-product property, $y' = 0$ when $y = 1, -2, -1$}. To express these roots as solutions to the differential equation, it's important that they follow this form:
$$ \boxed{y_1(t) = 1 \quad;\quad y_2(t) = -2 \quad;\quad y_3(t) = -1}$$
\textbf{(c)} We can draw a phase line for the differential equation by labelling the critical points (the roots) on a number line and performing a sign diagram analysis by choosing any $y$ value around the critical points and determining the sign of $y^{\prime}$:\\
$$ ---(-2)-----(-1)------(1)--- $$
Test any $y$ value in the interval $(-\infty, -2)$. Let's pick $y=-3$:
$$ (-\infty, -2): \quad\overset{\text{sub }y=-3\text{ into }y^{\prime}}\rightarrow \quad y^{\prime} = 6(-3-1)(-3 + 2)(-3+1) \quad\rightarrow ... $$
$$ ...\rightarrow \quad y^{\prime} = -48. $$
We only care about the sign of the answer. In this case, it's negative which means that the derivative is negative everywhere in this interval. Let's continue with the analysis for the rest of the intervals:
$$ (-2,-1): \quad\overset{\text{sub }y=-1.5\text{ into }y^{\prime}}\rightarrow \quad y^{\prime} = 6(-1.5-1)(-1.5+2)(-1.5+1) \quad\rightarrow \quad y^{\prime}=+3.75.$$
$$ (-1,1): \quad\overset{\text{sub }y=0.5\text{ into }y^{\prime}}\rightarrow \quad y^{\prime} = 6(0.5-1)(0.5+2)(0.5+1) \quad\rightarrow \quad y^{\prime} = -11.25.$$
$$ (1,\infty): \quad\overset{\text{sub }y=2\text{ into }y^{\prime}}\rightarrow \quad y^{\prime} = 6(2-1)(2+2)(2+1) \quad\rightarrow \quad y^{\prime} = +72. $$
With our signs, let's make our sign diagram on top of our number line:
$$ ---(-2)-----(-1)------(1)---  $$
$$ - \quad\quad\quad\quad\quad + \quad\quad\quad\quad\quad\quad\quad - \quad\quad\quad\quad\quad\quad + $$
The $y$-axis is the horizontal so when the derivative sign is negative, it will point towards the decreasing $y$ direction. If the derivative sign is positive, it will point towards the increasing $y$ direction:
$$ -<--(-2)-->--(-1)--<--(1)-->- $$
\pagebreak

\textbf{(d)} From the diagram above, we see that:
$$ \boxed{y_1(t) =1 \text{ is a source, which is unstable.}} $$
$$ \boxed{y_2(t) =-2 \text{ is a source, which is unstable.}} $$

$$\boxed{y_3(t) =-1 \text{ is a sink, which is stable.}} $$

\textbf{(e)} We have an initial condition that is $-\frac{3}{4}$ at $t=0$. Let's think about what would happen if we ran the clock forward and followed our arrows. If we were to place this point on our number line and let it move with the arrows, it would be between $y=-1$ and $y=1$. The arrow pushes it towards the stable point of $y=-1$ where it stays forever. In math:
$$ \boxed{\text{For $y(0)=-\frac{3}{4}$}\quad :\quad \lim_{t\to \infty} y(t) = -1} $$
Now, if we were to run the clock backwards, we would place our point $y=-\frac{3}{4}$ on the number line again, but this time we will ask where it came from. This is when it helps to think of unstable equilibria as sources. We see from the diagram that the source of $y=-\frac{3}{4}$ is $y=1$. Putting this in math:
$$ \boxed{\text{For $y(0)=-\frac{3}{4}$}\quad:\quad \lim_{t\to -\infty} y(t) = 1} $$



\end{document}