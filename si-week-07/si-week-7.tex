%%%%%%%%%%%%%%%%%%%%%%%%%%%%%%%%%%%%%%%%%%%%%%%%
% 1. document class
%%%%%%%%%%%%%%%%%%%%%%%%%%%%%%%%%%%%%%%%%%%%%%%%
 
\documentclass[a4paper,12pt]{article} 

%%%%%%%%%%%%%%%%%%%%%%%%%%%%%%%%%%%%%%%%%%%%%%%%
% 2. packages
%%%%%%%%%%%%%%%%%%%%%%%%%%%%%%%%%%%%%%%%%%%%%%%%

\usepackage[top = 2.5cm, bottom = 2.5cm, left = 2.5cm, right = 2.5cm]{geometry} 

\usepackage[T1]{fontenc}
\usepackage[utf8]{inputenc}
\usepackage{amsmath}
\usepackage{cancel}
\usepackage{amssymb}
\usepackage{multirow}
\usepackage{booktabs}
\usepackage{graphicx} 
\usepackage{setspace}
\setlength{\parindent}{0in}
\usepackage{float}
\usepackage{fancyhdr}
\usepackage[colorlinks=true,linkcolor=blue]{hyperref}
\usepackage{amsmath}
\usepackage{tikz}
\usetikzlibrary{tikzmark}

%%%%%%%%%%%%%%%%%%%%%%%%%%%%%%%%%%%%%%%%%%%%%%%%
% 3. header (and footer)
%%%%%%%%%%%%%%%%%%%%%%%%%%%%%%%%%%%%%%%%%%%%%%%%

\pagestyle{fancy}
\fancyhf{}

\lhead{\footnotesize }
\rhead{\footnotesize Collins \thepage}
\cfoot{\footnotesize} 

%%%%%%%%%%%%%%%%%%%%%%%%%%%%%%%%%%%%%%%%%%%%%%%%
% 4. the document
%%%%%%%%%%%%%%%%%%%%%%%%%%%%%%%%%%%%%%%%%%%%%%%%

\begin{document}

%%%%%%%%%%%%%%%%%%%%%%%%%%%%%%%%%%%%%%%%%%%%%%%%
% title section of the document
%%%%%%%%%%%%%%%%%%%%%%%%%%%%%%%%%%%%%%%%%%%%%%%%

\thispagestyle{empty}

\begin{tabular}{p{15.5cm}}
\\ Collin Collins \\
MATH 3400\\
SI Session 7 Practice Problems and Solutions\\
19 March 2024 \\
\hline

\end{tabular} 

\subsection*{Problem 1} 
Find the general solution using Variation of Parameters.
$$ y''-4y' = e^{ex}. $$
\\
 
Try it out before looking at the solution
\pagebreak

\subsubsection*{Solution to Problem 1:}
Taking a look at the form of our differential equation, we have what appears to be a 2nd order linear, non-homogenous, differential equation. So far we know that for non-homogenous, 2nd order, linear, differential equations, we can use the method of Undetermined Coefficients or Variation of Parameters. We are told to use the latter, so let's begin.\\

We'll remember some important information for this method.\\

I'm going to assume that it is clear that $\mu_{1,2}$ and $y_{1,2}$ along with any composition of these two are all functions of our independent variable, $x$.\\

With this assumption, we can write what we need to remember in a concise and easy to remember way.
\begin{equation}
	y(x) = y_c + y_p. \label{eq:1}
\end{equation}
\begin{equation}
	y_c = C_1y_1 + C_2y_2  \quad \text{and}\quad y_p = \mu_1y_1 + \mu_2y_2. \label{eq:2}
\end{equation}
\begin{equation}
	\mu_1 = -\int \frac{y_2f(x)}{\mathrm{W}[y_1,y_2]}dx \quad\text{and} \quad \mu_2 = \int \frac{y_1f(x)}{\mathrm{W}[y_1,y_2]}dx. \label{eq:3}
\end{equation}
One quick note. $\mathrm{W}[y_1,y_2]$ is the Wronskian of $y_1$ and $y_2$. It looks like this:
\begin{equation}
	\mathrm{W}[y_1,y_2] = \left|\begin{matrix}
	y_1 & y_2 \\
	y_1' & y_2'
\end{matrix}\right| = y_1y_2' - y_1'y_2. \label{eq:4}
\end{equation}
Okay, let's put everything in one equation:
$$ \boxed{y(x) = C_1y_1 + C_2y_2 -y_1\int \frac{y_2f(x)}{\mathrm{W}[y_1,y_2]}dx + y_2 \int \frac{y_1f(x)}{\mathrm{W}[y_1,y_2]}dx.} $$
Now, this is a lot. We shouldn't try to memorize this whole thing.\\

Let's go through solving the problem that we were given to familiarize ourselves with Equations (\ref{eq:1}-\ref{eq:4})\\

We will start by finding the complementary solution, $y_c=C_1 y_1+C_2 y_2$ by solving our differential equation as if $f(x)=0$.
$$ y''-4y' = 0. $$
Finding our discriminant:
$$ D = (-4)^2 - 4(1)(0) \rightarrow D = 16 \rightarrow D>0. $$
Since $D>0$ our complimentary solution will be of the form:
$$ y_c = C_1e^{\lambda_1x} + C_2e^{\lambda_2x} \quad\text{where}\quad \lambda_{1,2} = \frac{-b \pm \sqrt{D}}{2a}. $$
$$ \lambda_{1,2} = \frac{4 \pm 4}{2} \rightarrow \lambda_1 = 4 \quad\text{and}\quad \lambda_2 = 0. $$
Plugging these values of $\lambda$ into our complimentary solution, we have:
$$ \boxed{y_c = C_1\underbrace{e^{4x}}_{y_1} + C_2\underbrace{(1)}_{y_2}} $$
Now that we have our complimentary solution, let's start finding our particular solution, $y_p$, which is:
$$ y_p = \mu_1y_1 + \mu_2y_2. $$
We can immediately plug in our values for $y_1$ and $y_2$:
$$ y_p = \mu_1e^{4x} + \mu_2. $$
From here, let's remember our expressions for $\mu_1$ and $\mu_2$:
$$ \mu_1 = -\int \frac{y_2f(x)}{\mathrm{W[y_1,y_2]}} \quad \text{and} \quad \mu_2 = \int \frac{y_1f(x)}{\mathrm{W}[y_1,y_2]} .$$
Let's find the Wronskian of $y_1$ and $y_2$:
$$ \mathrm{W}[y_1,y_2] = \left|\begin{matrix}
	y_1 & y_2 \\
	y_1' & y_2'
\end{matrix}\right| \rightarrow \mathrm{W}\left[e^{4x}, 1\right] = \left|\begin{matrix}
	e^{4x} & 1 \\
	4e^{4x} & 0
\end{matrix}\right| = e^{4t}(0) - 4e^{4x}(1) \rightarrow \boxed{\mathrm{W}\left[e^{4x}, 1\right] = -4e^{4x}}$$
If $f(x) = e^{ex}$, then we can substitute all of our known values into the integrals for $\mu_1$ and $\mu_2$.
$$ \mu_1 = -\int \frac{e^{ex}}{-4e^{4x}}dx \quad\text{and}\quad \mu_2 = \int \frac{e^{4x}e^{ex}}{-4e^{4x}}dx.$$
Let's use exponent rules and pull out constants to simplify these integrals:
$$ \mu_1 = \frac{1}{4} \int e^{ex - 4x}dx \quad\text{and}\quad \mu_2 = -\frac{1}{4}\int e^{4x + ex - 4x} dx. $$
Simplifying our exponents further, we have:
$$ \mu_1 = \frac{1}{4}\int e^{(e-4)x}dx \quad\text{and}\quad \mu_2 = -\frac{1}{4}\int e^{ex}dx.  $$
Finally, evaluating these integrals, we have:
$$ \mu_1 = \frac{1}{\underbrace{4e-16}_{\text{came from }4(e-4)}}e^{(e-4)x} \quad\text{and}\quad \mu_2 = -\frac{1}{4e} e^{ex}  $$
We can take the almost complete particular solution that we had earlier: $y_p = \mu_1 e^{4x} + \mu_2$, and plug in our values for $\mu_{1,2}$.
$$ y_p = \left[\frac{e^{(e-4)x}}{4e-16}\right]e^{4x} - \frac{e^{ex}}{4e} $$
Simplifying the first term,
$$ \boxed{y_p = \frac{e^{ex}}{4e-16} - \frac{e^{ex}}{4e}} $$
We can stop here if we'd like. There's no need to get any more simplified than this, unless we have a good reason to.\\

At this point, we will remember Equation (\ref{eq:1}):
$$ y(x) = y_c + y_p. $$
With our complementary and particular solutions, we can write our general solution.
$$ \underline{\boxed{y_g(x) = \underbrace{C_1e^{4x} + C_2}_{y_c} + \underbrace{\frac{e^{ex}}{4e-16} - \frac{e^{ex}}{4e}}_{y_p}}} $$

\end{document}