%%%%%%%%%%%%%%%%%%%%%%%%%%%%%%%%%%%%%%%%%%%%%%%%
% 1. Document Class
%%%%%%%%%%%%%%%%%%%%%%%%%%%%%%%%%%%%%%%%%%%%%%%%
 
 % The first command you will always have will declare your document class. This tells LaTeX what type of document you are creating (article, presentation, poster, etc). 
% \documentclass is the command
% in {} you specify the type of document
% in [] you define additional parameters
 
\documentclass[a4paper,12pt]{article} % This defines the style of your paper

% We usually use the article type. The additional parameters are the format of the paper you want to print it on and the standard font size. For us this is a4paper and 12pt.

%%%%%%%%%%%%%%%%%%%%%%%%%%%%%%%%%%%%%%%%%%%%%%%%
% 2. Packages
%%%%%%%%%%%%%%%%%%%%%%%%%%%%%%%%%%%%%%%%%%%%%%%%

% Packages are libraries of commands that LaTeX can call when compiling the document. With the specialized commands you can customize the formatting of your document.
% If the packages we call are not installed yet, TeXworks will ask you to install the necessary packages while compiling.

% First, we usually want to set the margins of our document. For this we use the package geometry. We call the package with the \usepackage command. The package goes in the {}, the parameters again go into the [].
\usepackage[top = 2.5cm, bottom = 2.5cm, left = 2.5cm, right = 2.5cm]{geometry} 

% Unfortunately, LaTeX has a hard time interpreting German Umlaute. The following two lines and packages should help. If it doesn't work for you please let me know.
\usepackage[T1]{fontenc}
\usepackage[utf8]{inputenc}
\usepackage{amsmath}
\usepackage{cancel}
\usepackage{amssymb}

\usepackage{pgfplots}


% The following two packages - multirow and booktabs - are needed to create nice looking tables.
\usepackage{multirow} % Multirow is for tables with multiple rows within one cell.
\usepackage{booktabs} % For even nicer tables.

% As we usually want to include some plots (.pdf files) we need a package for that.
\usepackage{graphicx} 

% The default setting of LaTeX is to indent new paragraphs. This is useful for articles. But not really nice for homework problem sets. The following command sets the indent to 0.
\usepackage{setspace}
\setlength{\parindent}{0in}

% Package to place figures where you want them.
\usepackage{float}

% The fancyhdr package let's us create nice headers.
\usepackage{fancyhdr}

% Citing references will be hyperlinked and blue
\usepackage[colorlinks=true,linkcolor=blue]{hyperref}

\usepackage{amsmath}

\usepackage{tikz}
\usetikzlibrary{tikzmark}

%%%%%%%%%%%%%%%%%%%%%%%%%%%%%%%%%%%%%%%%%%%%%%%%
% 3. Header (and Footer)
%%%%%%%%%%%%%%%%%%%%%%%%%%%%%%%%%%%%%%%%%%%%%%%%

% To make our document nice we want a header and number the pages in the footer.

\pagestyle{fancy} % With this command we can customize the header style.

\fancyhf{} % This makes sure we do not have other information in our header or footer.

\lhead{\footnotesize }% \lhead puts text in the top left corner. \footnotesize sets our font to a smaller size.

%\rhead works just like \lhead (you can also use \chead)
\rhead{\footnotesize Collins \thepage} %<---- Fill in your lastnames.

% Similar commands work for the footer (\lfoot, \cfoot and \rfoot).
% We want to put our page number in the center.
\cfoot{\footnotesize} 


%%%%%%%%%%%%%%%%%%%%%%%%%%%%%%%%%%%%%%%%%%%%%%%%
% 4. Your document
%%%%%%%%%%%%%%%%%%%%%%%%%%%%%%%%%%%%%%%%%%%%%%%%

% Now, you need to tell LaTeX where your document starts. We do this with the \begin{document} command.
% Like brackets every \begin{} command needs a corresponding \end{} command. We come back to this later.

\begin{document}


%%%%%%%%%%%%%%%%%%%%%%%%%%%%%%%%%%%%%%%%%%%%%%%%
%%%%%%%%%%%%%%%%%%%%%%%%%%%%%%%%%%%%%%%%%%%%%%%%

%%%%%%%%%%%%%%%%%%%%%%%%%%%%%%%%%%%%%%%%%%%%%%%%
% Title section of the document
%%%%%%%%%%%%%%%%%%%%%%%%%%%%%%%%%%%%%%%%%%%%%%%%

% For the title section we want to reproduce the title section of the Problem Set and add your names.

\thispagestyle{empty} % This command disables the header on the first page. 

\begin{tabular}{p{15.5cm}} % This is a simple tabular environment to align your text nicely 
\\ Collin Collins \\
MATH 3400\\
SI Session 9 Practice Problems and Solutions\\
8 April 2024 \\
\hline % \hline produces horizontal lines.

\end{tabular} 

\subsection*{Problem 1} Compute the power series solution for the differential equation about the ordinary point, $x_{0}=0$:
$$y^{\prime \prime}+2 x y^{\prime}+2 y=0. $$
\\
 
Try it out before looking at the solution.
\pagebreak
 
 \subsubsection*{Solution to Problem 1:}
 To begin, we will write that the final form of the power series of our solution is the sum of two linearly independent solutions:
 $$ y(x) = y_1(x) + y_2(x). $$
Much like other techniques, we will begin with a guess of our solution. This time the guess is the simple power series:
$$ y_{\text{guess}}(x) = \sum_{n=0}^{\infty} c_nx^n. $$

We can differentiate and integrate a power series term-by-term, so let's do that for our guess in order to write our differential equation as a sum of power series.
$$ y_{\text{guess}}'(x) = \frac{d}{dx}\left[\sum_{n=0}^{\infty} c_nx^n\right] \quad\implies\quad y_{\text{guess}}'(x) = \sum_{n=1}^{\infty} (n)c_nx^{n-1}. $$
We begin our derivative's series at $n=1$ because the 0$^{\text{th}}$ term is 0:
$$ y_{\text{guess}}(x) = \underbrace{c_0}_{n=0} + \underbrace{c_1x}_{n=1} + \underbrace{c_2x^2}_{n=2}+\ldots $$
$$ y_{\text{guess}}'(x) = \underbrace{0}_{n=0} + \underbrace{c_1}_{n=1} + \underbrace{2c_2x}_{n=2} +\ldots $$

Next, we find the second derivative of our solution form (which will begin at $n=2$).
$$ y_{\text{guess}}''(x) = \frac{d}{dx}\left[\sum_{n=1}^{\infty} (n)c_nx^{n-1}\right] \quad\implies\quad y_{\text{guess}}''(x)= \sum_{n=2}^{\infty} (n-1)(n)c_nx^{n-2}. $$
We now have the following representations of our guess and their derivatives:
$$ y_{\text{guess}}(x) = \sum_{n=0}^{\infty} c_nx^n. $$
$$ y_{\text{guess}}'(x) = \sum_{n=1}^{\infty} (n)c_nx^{n-1}. $$
$$ y_{\text{guess}}''(x)= \sum_{n=2}^{\infty} (n-1)(n)c_nx^{n-2}. $$
Let's plug these into our differential equation: $y^{\prime \prime}+2 x y^{\prime}+2 y=0.$
$$ \left[\sum_{n=2}^{\infty} (n-1)(n)c_nx^{n-2}\right] + 2x\left[\sum_{n=1}^{\infty} (n)c_nx^{n-1}\right] + 2\left[\sum_{n=0}^{\infty} c_nx^n\right]=0. $$
\pagebreak

Our goal is to be able to combine these series into one big series and factor out a common term of $x^n$. To do that, we will need to make sure all series:

\begin{enumerate}
	\item Have a factor of $x^n$.
	\item Begin at the same index
\end{enumerate}

$\boxed{\underline{\text{IMPORTANT NOTE}}}:$\\

 Before we reindex any of our power series, we want to plug them into our differential equation. Why? Because we want to distribute terms like the $2x$ into our series before we reindex, or else we'll have to reindex twice (which is not fun).\\
 
 Let's begin with the left-most term of our differential equation, $y''(x)$.\\
 
Since we don't distribute anything to this power series, we can reindex immediately:
$$ \text{Let }k = n-2 \text{, implying } n = k+2. $$
$$ \sum_{n=2}^{\infty} (n-1)(n)c_nx^{n-2} \quad\rightarrow\quad \sum_{k=0}^{\infty} ((k+2)-1)(k+2)c_{k+2}x^{k} \quad\implies\quad \sum_{k=0}^{\infty} (k+1)(k+2)c_{k+2}x^{k}.$$
To show that I've already reindexed, I prefer to keep the index as $k$. We now have:
$$ y_{\text{guess}}''(x) = \sum_{k=0}^{\infty} (k+1)(k+2)c_{k+2}x^{k} $$
Now, let's do the same, but with the second term of our differential equation.\\
 
 We will distribute $2x$ into the power series representation of $y'(x)$:
 $$ 2x\left[\sum_{n=1}^{\infty} (n)c_nx^{n-1}\right] \quad\rightarrow\quad \sum_{n=1}^{\infty} 2(n)c_nx^{n-1 + 1} \quad\rightarrow\quad \sum_{n=1}^{\infty} 2(n)c_nx^{n}. $$
 Since this power series now contains a factor of $x^n$, let's reindex $n$ to be $k$:
 $$ 2xy_{\text{guess}}'(x) = \sum_{k=1}^{\infty} 2(k)c_kx^{k}$$ 
 Let's work with the last term of our series-expanded differential equation:
 $$ 2\left[\sum_{n=0}^{\infty} c_nx^n\right] \quad\rightarrow\quad \sum_{n=0}^{\infty} 2c_nx^n. $$ 
 This term already contains a factor of $x^n$, so let's let $n = k$ and write our last distributed term out:
 $$ 2y_{\text{guess}}(x) =  \sum_{k=0}^{\infty} 2c_kx^k $$
 
 At this point we have all of our power series containing a factor of $x^k$. When we put them together we have:
 $$ y'' + 2xy' + 2y = \left[\sum_{k=0}^{\infty} (k+1)(k+2)c_{k+2}x^{k}\right] + \left[\sum_{k=1}^{\infty} 2(k)c_kx^{k}\right] + \left[\sum_{k=0}^{\infty} 2c_kx^k\right] = 0.  $$
 Notice that we are one step away from making one big $\sum$. We need all of our series to start at $k=1$.\\
 
 To do this, we will pull the 1$^\text{st}$ terms out of our first and last series.\\
 
 Pulling the first term out of $\sum_{k=0}^{\infty} (k+1)(k+2)c_{k+2}x^{k}$, we have:
 $$\left[\sum_{k=0}^{\infty} (k+1)(k+2)c_{k+2}x^{k}\right] \quad\implies\quad \underbrace{2c_2}_{0^{\text{th}}\text{ term}} + \left[\sum_{k=1}^{\infty} (k+1)(k+2)c_{k+2}x^{k}\right]. $$
 Pulling the first term out of $\sum_{k=0}^{\infty} 2c_kx^k$, we have:
 $$ \left[\sum_{k=0}^{\infty} 2c_kx^k\right] \quad\implies\quad  \underbrace{2c_0}_{0^{\text{th}}\text{ term}} + \left[\sum_{k=1}^{\infty} 2c_kx^k\right]. $$
 Rewriting our differential equation:
 $$ 2c_2 + \left[\sum_{k=1}^{\infty} (k+1)(k+2)c_{k+2}x^{k}\right] + \left[\sum_{k=1}^{\infty} 2(k)c_kx^{k}\right] +  2c_0 + \left[\sum_{k=1}^{\infty} 2c_kx^k\right] = 0. $$
 We will combine our sums since they begin at the same index and we will factor out $x^k$ since all of them have that term in common.
 $$ \underbrace{2(c_2 + c_0)}_{\text{this part = }0} + \sum_{k=1}^{\infty} \underbrace{\bigg[(k+1)(k+2)c_{k+2} + 2c_k(k + 1)\bigg]}_{\text{this part also = }0}x^k = 0 $$
Let's set our coefficients to be equal to zero. Here the coefficients outside of the sum are going to be equal to zero when $k=0$ and the coefficients inside the sum are going to be equal to zero when $k=1,2,3\ldots$
 
 $$ \begin{array}{rlrl}
k & =0 & 2(c_2 + c_0) & =0 \\
k & =1,2,3, \ldots & (k+1)(k+2) c_{k+2}+2c_{k}(k+1) & =0
\end{array} $$
This means that
$$ 2(c_2 + c_0) = 0 \quad\implies\quad \boxed{c_2 = -c_0} $$
and
$$ (k+1)(k+2) c_{k+2}+2c_{k}(k+1)=0 \quad\implies\quad c_{k+2} = -\frac{2c_k(k+1)}{(k+1)(k+2)} \quad\implies\ldots $$
$$ \ldots\quad\implies\quad c_{k+2} = \boxed{-\frac{2c_k}{(k+2)}} $$

Let's use the boxed relations to write out some of the first few coefficients, starting at $k=1$.
 $$ \begin{array}{rl}
k=1: & c_3 = -\frac{2c_1}{3} \quad\overset{\text{rewriting}}\implies\quad c_3 = -\frac{2}{3\times1}c_1\\

k=2: & c_4 = -\frac{2c_2}{4} \quad\overset{c_2=-c_0}\implies\quad c_4 = \frac{2}{4}c_0 \\

k=3: & c_5 = -\frac{2c_3}{5} \quad\overset{\text{using }c_3}\implies\quad c_5 = \frac{2\times2}{5\times3\times1}c_1 \\

k=4: & c_6 = -\frac{2c_4}{6} \quad\overset{\text{using }c_4}\implies\quad c_6 = -\frac{2\times2}{6\times4}c_0 \\

k=5: & c_7 = -\frac{2c_5}{7} \quad\overset{\text{using }c_5}\implies\quad c_7=-\frac{2\times2\times2}{7\times5\times3\times1}c_1 \\

k=6: & c_8 = -\frac{2c_6}{8} \quad\overset{\text{using }c_6}\implies\quad c_8 = \frac{2\times2\times2}{8\times6\times4}c_0 \\

k=7: & c_9 = -\frac{2c_7}{9} \quad\overset{\text{using }c_7}\implies\quad c_9 = \frac{2\times2\times2\times2}{9\times7\times5\times3\times1}c_1 \\

k=8: & c_{10} = -\frac{2c_8}{10} \quad\overset{\text{using }c_8}\implies\quad c_{10} = -\frac{2\times2\times2\times2}{10\times8\times6\times4}c_0 \\
k=9: & c_{11} = -\frac{2c_9}{11} \quad\overset{\text{using }c_9}\implies\quad c_{11} = -\frac{2\times2\times2\times2\times2}{11\times9\times7\times5\times3\times1}c_1
\end{array} $$


It's time for a reality check. If things are going right, we should see that all of our coefficients are in terms of two coefficients. In this case, $c_0$ and $c_1$. If you don't end up with something like this, make sure you've simplified completely.\\

 Let's split things up into $c_0$'s and $c_1$'s to make the pattern more clear.\\

The $c_1$ terms are below. $p$ is a dummy index that will help us determine the parts of our sequence that define the pattern of our odd-numbered terms.
$$
\begin{array}{rl}
p=1 :& c_3 = -\frac{2}{3\times1}c_1\\
p=2 :& c_5 = \frac{2\times2}{5\times3\times1}c_1\\
p=3 :& c_7=-\frac{2\times2\times2}{7\times5\times3\times1}c_1\\
p=4 :& c_9 = \frac{2\times2\times2\times2}{9\times7\times5\times3\times1}c_1\\
p=5 :&  c_{11} = -\frac{2\times2\times2\times2\times2}{11\times9\times7\times5\times3\times1}c_1\\
\end{array}
$$
We'll start off with some rules for $c_1$.
\begin{enumerate}
	\item Subscript starts at 3 and iterates over odd numbers, so our coefficient is defined in terms of $p$ as $c_{2p+1}$.
	\item Sign starts $(-1)$ for $c_3$ and oscillates from $(-1)$ to $(+1)$, so that portion of our coefficients' recurrence relation is $(-1)^p$.
	\item Numerator has a 2 raised to a power of 1 for $c_3$ and 5 for $c_{11}$, so the numerator follows the patter of $2^p$.
	\item Denominator has an odd number factorial defined by the constant's subscript, this is defined using $(2p+1)!!$, which will be discussed in a moment.
\end{enumerate}
Let's write the sequence that satisfies these rules and explain each part after:
$$ \underline{c_{2p+1} = (-1)^{p}\left[\frac{2^{p}}{(2p+1)!!}\right]c_1.}$$
The double factorial notation was invented in 1902 by physicist Arthur Schuster, and we should be glad that someone came up with it because it makes our work much more simple. In short, it's a factorial that increments down by two numbers instead of one. \\

Feel free to check that this sequence makes sense. Plug in some values of $p$ to see if the results matches our $c_{2p+1}$ sequence.\\

The sequence that we have written is a formal and compact way of representing all odd-numbered coefficients, $c_{2p+1}$ of the terms of our differential equation solution. The odd-numbered coefficients correspond to the odd-numbered powers of $x$: $x^{2p+1}$. This supports the notion that our solution will be the combination of two linearly independent solutions $y_1(x)$ and $y_2(x)$.\\

 It makes no difference which of the two previously mentions halves of the solution the $c_{2p+1}$ terms correspond to, let's arbitrarily pick them to be for the $y_2(x)$ half of our final solution.\\

Remembering the form of our solution guess:
$$ y_{\text{guess}}(x) = \sum_{n=0}^{\infty} c_{n}x^n, $$
The odd-numbered terms will be of the form:
$$ y_2(x) = c_1 \sum_{p=0}^{\infty} c_{2p+1}x^{2p+1}. $$
Which can be fully written as:
$$ \boxed{y_2(x) = c_1\sum_{p=0}^{\infty} (-1)^p \left[\frac{2^p}{(2p+1)!!}\right]x^{2p+1}} $$

Something to note: our solution begins at $p=0$, which is how we obtain the $x^{2(0)+1} \implies x$ term of our solution.\\

Now, let's examine the $c_0$ terms.
$$
\begin{array}{rl}
p=1: & c_2 = -\frac{2^0}{1} c_0. \\
p=2: & c_4 = \frac{2}{4}c_0, \\
p=3: & c_6 = -\frac{2\times2}{6\times4}c_0, \\
p=4: & c_8 = \frac{2\times2\times2}{8\times6\times4}c_0, \\
p=5: & c_{10} = -\frac{2\times2\times2\times2}{10\times8\times6\times4}c_0. \\
\end{array}
$$

We'll start off with some rules for $c_0$.
\begin{enumerate}
	\item Subscript starts at 2 and iterates over even numbers, so our coefficients follow the rule $c_{2p}$.
	\item Sign starts at $(-1)$ for $c_2$ and oscillates from $(-1)$ to $(+1)$, so we have a factor of $(-1)^p$.
	\item Numerator has a 2 raised to a power of 0 for $c_2$ and 4 for $c_{10}$, the rule that governs this behavior is $2^{p-1}$.
	\item Denominator has an even number factorial defined by the constant's subscript, but it's missing the 2 term, yielding a factor of $\frac{2}{(2p)!!}$.
\end{enumerate}
Let's write the sequence that satisfies these rules:
$$ c_{2p} = (-1)^{p}\left[\frac{2\left(2^{p-1}\right)}{(2p)!!}\right]c_0.$$
Notice that we can distribue the 2 to the $2^{p-1}$ term, giving us:
$$ \underline{c_{2p} = (-1)^{p}\left[\frac{2^{p}}{(2p)!!}\right]c_0} $$

We now have the recurrence relation for the even-numbered, $c_{2p}$, coefficients. These coefficients belong to the even-powered $x$ terms: $x^{2p}$. This half of our final solution will be called $y_1(x)$. Similar to the $y_2(x)$ half of our solution, it will be of the form:

$$ y_1(x) = c_0 \sum_{p=0}^{\infty} c_{2p}x^{2p}. $$

In its full form:
$$ \boxed{y_1(x) = c_0 \sum_{p=0}^{\infty} (-1)^{p}\left[\frac{2^{p}}{(2p)!!}\right] x^{2p}} $$
Our final solution is then:
$$ y(x) = y_1(x) + y_2(x) \quad\implies\quad y(x) = \sum_{p=0}^{\infty} c_0c_{2p}x^{2p} + c_1c_{2p+1}x^{2p+1} \quad\implies\ldots $$
$$ \underline{\boxed{y(x) = \sum_{p=0}^{\infty} c_0\left( (-1)^{p}\left[\frac{2^{p}}{(2p)!!}\right]\right)x^{2p} + c_1\left((-1)^{p}\left[\frac{2^{p}}{(2p+1)!!}\right]\right)x^{2p+1}}}$$

Writing some terms out:
$$ y(x) = c_0 + c_1x -c_0x^2 - c_1\frac{2}{3}x^3 + c_0\frac{1}{2}x^4 + c_1\frac{4}{15}x^5-c_0\frac{1}{6}x^6-c_1\frac{8}{105}x^7+... $$

\end{document}