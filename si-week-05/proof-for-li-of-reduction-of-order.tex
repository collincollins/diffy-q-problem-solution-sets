%%%%%%%%%%%%%%%%%%%%%%%%%%%%%%%%%%%%%%%%%%%%%%%%
% 1. document class
%%%%%%%%%%%%%%%%%%%%%%%%%%%%%%%%%%%%%%%%%%%%%%%%
 
\documentclass[a4paper,12pt]{article} 

%%%%%%%%%%%%%%%%%%%%%%%%%%%%%%%%%%%%%%%%%%%%%%%%
% 2. packages
%%%%%%%%%%%%%%%%%%%%%%%%%%%%%%%%%%%%%%%%%%%%%%%%

\usepackage[top = 2.5cm, bottom = 2.5cm, left = 2.5cm, right = 2.5cm]{geometry} 

\usepackage[T1]{fontenc}
\usepackage[utf8]{inputenc}
\usepackage{amsmath}
\usepackage{cancel}
\usepackage{amssymb}
\usepackage{multirow}
\usepackage{booktabs}
\usepackage{graphicx} 
\usepackage{setspace}
\setlength{\parindent}{0in}
\usepackage{float}
\usepackage{fancyhdr}
\usepackage[colorlinks=true,linkcolor=blue]{hyperref}
\usepackage{amsmath}
\usepackage{tikz}
\usetikzlibrary{tikzmark}

%%%%%%%%%%%%%%%%%%%%%%%%%%%%%%%%%%%%%%%%%%%%%%%%
% 3. header (and footer)
%%%%%%%%%%%%%%%%%%%%%%%%%%%%%%%%%%%%%%%%%%%%%%%%

\pagestyle{fancy}
\fancyhf{}

\lhead{\footnotesize }
\rhead{\footnotesize Collins \thepage}
\cfoot{\footnotesize} 

%%%%%%%%%%%%%%%%%%%%%%%%%%%%%%%%%%%%%%%%%%%%%%%%
% 4. the document
%%%%%%%%%%%%%%%%%%%%%%%%%%%%%%%%%%%%%%%%%%%%%%%%

\begin{document}

%%%%%%%%%%%%%%%%%%%%%%%%%%%%%%%%%%%%%%%%%%%%%%%%
% title section of the document
%%%%%%%%%%%%%%%%%%%%%%%%%%%%%%%%%%%%%%%%%%%%%%%%

\thispagestyle{empty}

\begin{tabular}{p{15.5cm}}
\\ Collin Collins \\
MATH 3400\\
26 February 2024 \\
\hline

\end{tabular}
\\
\\

\textbf{\textit{Theorem}}: Consider the linear, homogeneous, second-order differential equation:
$$
y^{\prime \prime}(x)+p(x) y^{\prime}(x)+q(x) y(x)=0\quad;\quad\forall x\in \mathbb{R}
$$
If $y_1(x)$ is a nontrivial solution, and $y_2(x)$ is defined as:
$$
y_2(x) \triangleq y_1(x) \int \frac{e^{-\int p(x) d x}}{y_1^2(x)} d x
$$
Then $y_1(x)$ and $y_2(x)$ are linearly independent; thus, the general solution can be expressed as:
$$
y_g(x)=C_1 y_1(x)+C_2 y_2(x)
$$

\textbf{\textit{Proof}}: The Wronskian of $y_1$ and $y_2$ is:
$$
\mathrm{W}\left[y_1, y_2\right]=\left|\begin{array}{ll}
y_1(x) & y_2(x) \\
y_1^{\prime}(x) & y_2^{\prime}(x)
\end{array}\right|=y_1(x) y_2^{\prime}(x)-y_2(x) y_1^{\prime}(x)
$$
Substituting the definition of $y_2(x)$ and differentiating:
$$ \operatorname{W}[y_1, y_2] = y_1(x)\left[y_1'(x)\int{\frac{e^{-\int p(x)dx}}{y_1^2(x)}}dx + y_1(x)\frac{e^{-\int p(x)dx}}{y_1^2(x)}\right] - \left[y_1(x)\int{\frac{e^{-\int p(x)dx}}{y_1^2(x)}}dx\right]y_1'(x)$$
Simplifying, we obtain:
$$ \operatorname{W}[y_1, y_2] = \frac{e^{-\int p(x)dx}}{y_1(x)}$$

For $y_1$ and $y_2$ to be linearly independent, their Wronskian must be nonzero. Hence,
$$
\frac{e^{-\int p(x) d x}}{y_1(x)} \neq 0 \quad \Longrightarrow \quad e^{-\int p(x) d x} \neq 0\quad;\quad\forall p(x): \int p(x) d x<+\infty.
$$.
\textbf{QED}




\end{document}