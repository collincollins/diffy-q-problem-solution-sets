%%%%%%%%%%%%%%%%%%%%%%%%%%%%%%%%%%%%%%%%%%%%%%%%
% 1. Document Class
%%%%%%%%%%%%%%%%%%%%%%%%%%%%%%%%%%%%%%%%%%%%%%%%
 
 % The first command you will always have will declare your document class. This tells LaTeX what type of document you are creating (article, presentation, poster, etc). 
% \documentclass is the command
% in {} you specify the type of document
% in [] you define additional parameters
 
\documentclass[a4paper,12pt]{article} % This defines the style of your paper

% We usually use the article type. The additional parameters are the format of the paper you want to print it on and the standard font size. For us this is a4paper and 12pt.

%%%%%%%%%%%%%%%%%%%%%%%%%%%%%%%%%%%%%%%%%%%%%%%%
% 2. Packages
%%%%%%%%%%%%%%%%%%%%%%%%%%%%%%%%%%%%%%%%%%%%%%%%

% Packages are libraries of commands that LaTeX can call when compiling the document. With the specialized commands you can customize the formatting of your document.
% If the packages we call are not installed yet, TeXworks will ask you to install the necessary packages while compiling.

% First, we usually want to set the margins of our document. For this we use the package geometry. We call the package with the \usepackage command. The package goes in the {}, the parameters again go into the [].
\usepackage[top = 2.5cm, bottom = 2.5cm, left = 2.5cm, right = 2.5cm]{geometry} 

% Unfortunately, LaTeX has a hard time interpreting German Umlaute. The following two lines and packages should help. If it doesn't work for you please let me know.
\usepackage[T1]{fontenc}
\usepackage[utf8]{inputenc}
\usepackage{amsmath}
\usepackage{cancel}
\usepackage{amssymb}
\usepackage{halloweenmath}

% The following two packages - multirow and booktabs - are needed to create nice looking tables.
\usepackage{multirow} % Multirow is for tables with multiple rows within one cell.
\usepackage{booktabs} % For even nicer tables.

% As we usually want to include some plots (.pdf files) we need a package for that.
\usepackage{graphicx} 

% The default setting of LaTeX is to indent new paragraphs. This is useful for articles. But not really nice for homework problem sets. The following command sets the indent to 0.
\usepackage{setspace}
\setlength{\parindent}{0in}

% Package to place figures where you want them.
\usepackage{float}

% The fancyhdr package let's us create nice headers.
\usepackage{fancyhdr}

% Citing references will be hyperlinked and blue
\usepackage[colorlinks=true,linkcolor=blue]{hyperref}

\usepackage{amsmath}

\usepackage{tikz}
\usetikzlibrary{tikzmark}

%%%%%%%%%%%%%%%%%%%%%%%%%%%%%%%%%%%%%%%%%%%%%%%%
% 3. Header (and Footer)
%%%%%%%%%%%%%%%%%%%%%%%%%%%%%%%%%%%%%%%%%%%%%%%%

% To make our document nice we want a header and number the pages in the footer.

\pagestyle{fancy} % With this command we can customize the header style.

\fancyhf{} % This makes sure we do not have other information in our header or footer.

\lhead{\footnotesize }% \lhead puts text in the top left corner. \footnotesize sets our font to a smaller size.

%\rhead works just like \lhead (you can also use \chead)
\rhead{\footnotesize Collins \thepage} %<---- Fill in your lastnames.

% Similar commands work for the footer (\lfoot, \cfoot and \rfoot).
% We want to put our page number in the center.
\cfoot{\footnotesize} 


%%%%%%%%%%%%%%%%%%%%%%%%%%%%%%%%%%%%%%%%%%%%%%%%
% 4. Your document
%%%%%%%%%%%%%%%%%%%%%%%%%%%%%%%%%%%%%%%%%%%%%%%%

% Now, you need to tell LaTeX where your document starts. We do this with the \begin{document} command.
% Like brackets every \begin{} command needs a corresponding \end{} command. We come back to this later.

\begin{document}


%%%%%%%%%%%%%%%%%%%%%%%%%%%%%%%%%%%%%%%%%%%%%%%%
%%%%%%%%%%%%%%%%%%%%%%%%%%%%%%%%%%%%%%%%%%%%%%%%

%%%%%%%%%%%%%%%%%%%%%%%%%%%%%%%%%%%%%%%%%%%%%%%%
% Title section of the document
%%%%%%%%%%%%%%%%%%%%%%%%%%%%%%%%%%%%%%%%%%%%%%%%

% For the title section we want to reproduce the title section of the Problem Set and add your names.

\thispagestyle{empty} % This command disables the header on the first page. 

\begin{tabular}{p{15.5cm}} % This is a simple tabular environment to align your text nicely 
\\ Collin Collins \\
MATH 3400\\
26 February 2024 \\
\hline % \hline produces horizontal lines.

\end{tabular} 
\\
\\

\textbf{\textit{Theorem}}: Consider the linear, homogeneous, second-order differential equation:
$$
y^{\prime \prime}(x)+p(x) y^{\prime}(x)+q(x) y(x)=0\quad;\quad\forall x\in \mathbb{R}
$$
If $y_1(x)$ is a nontrivial solution, and $y_2(x)$ is defined as:
$$
y_2(x) \triangleq y_1(x) \int \frac{e^{-\int p(x) d x}}{y_1^2(x)} d x
$$
Then $y_1(x)$ and $y_2(x)$ are linearly independent; thus, the general solution can be expressed as:
$$
y_g(x)=C_1 y_1(x)+C_2 y_2(x)
$$

\textbf{\textit{Proof}}: The Wronskian of $y_1$ and $y_2$ is:
$$
\mathrm{W}\left[y_1, y_2\right]=\left|\begin{array}{ll}
y_1(x) & y_2(x) \\
y_1^{\prime}(x) & y_2^{\prime}(x)
\end{array}\right|=y_1(x) y_2^{\prime}(x)-y_2(x) y_1^{\prime}(x)
$$
Substituting the definition of $y_2(x)$ and differentiating:
$$ \operatorname{W}[y_1, y_2] = y_1(x)\left[y_1'(x)\int{\frac{e^{-\int p(x)dx}}{y_1^2(x)}}dx + y_1(x)\frac{e^{-\int p(x)dx}}{y_1^2(x)}\right] - \left[y_1(x)\int{\frac{e^{-\int p(x)dx}}{y_1^2(x)}}dx\right]y_1'(x)$$
Simplifying, we obtain:
$$ \operatorname{W}[y_1, y_2] = \frac{e^{-\int p(x)dx}}{y_1(x)}$$

For $y_1$ and $y_2$ to be linearly independent, their Wronskian must be nonzero. Hence,
$$
\frac{e^{-\int p(x) d x}}{y_1(x)} \neq 0 \quad \Longrightarrow \quad e^{-\int p(x) d x} \neq 0\quad;\quad\forall p(x): \int p(x) d x<+\infty.
$$.
\textbf{QED}




\end{document}