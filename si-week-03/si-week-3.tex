%%%%%%%%%%%%%%%%%%%%%%%%%%%%%%%%%%%%%%%%%%%%%%%%
% 1. document class
%%%%%%%%%%%%%%%%%%%%%%%%%%%%%%%%%%%%%%%%%%%%%%%%
 
\documentclass[a4paper,12pt]{article} 

%%%%%%%%%%%%%%%%%%%%%%%%%%%%%%%%%%%%%%%%%%%%%%%%
% 2. packages
%%%%%%%%%%%%%%%%%%%%%%%%%%%%%%%%%%%%%%%%%%%%%%%%

\usepackage[top = 2.5cm, bottom = 2.5cm, left = 2.5cm, right = 2.5cm]{geometry} 

\usepackage[T1]{fontenc}
\usepackage[utf8]{inputenc}
\usepackage{amsmath}
\usepackage{cancel}
\usepackage{amssymb}
\usepackage{multirow}
\usepackage{booktabs}
\usepackage{graphicx} 
\usepackage{setspace}
\setlength{\parindent}{0in}
\usepackage{float}
\usepackage{fancyhdr}
\usepackage[colorlinks=true,linkcolor=blue]{hyperref}
\usepackage{amsmath}
\usepackage{tikz}
\usetikzlibrary{tikzmark}

%%%%%%%%%%%%%%%%%%%%%%%%%%%%%%%%%%%%%%%%%%%%%%%%
% 3. header (and footer)
%%%%%%%%%%%%%%%%%%%%%%%%%%%%%%%%%%%%%%%%%%%%%%%%

\pagestyle{fancy}
\fancyhf{}

\lhead{\footnotesize }
\rhead{\footnotesize Collins \thepage}
\cfoot{\footnotesize} 

%%%%%%%%%%%%%%%%%%%%%%%%%%%%%%%%%%%%%%%%%%%%%%%%
% 4. the document
%%%%%%%%%%%%%%%%%%%%%%%%%%%%%%%%%%%%%%%%%%%%%%%%

\begin{document}

%%%%%%%%%%%%%%%%%%%%%%%%%%%%%%%%%%%%%%%%%%%%%%%%
% title section of the document
%%%%%%%%%%%%%%%%%%%%%%%%%%%%%%%%%%%%%%%%%%%%%%%%

\thispagestyle{empty}

\begin{tabular}{p{15.5cm}}
\\ Collin Collins \\
MATH 3400\\
SI Session 3 Practice Problems and Solutions\\
10 February 2023 \\
\hline

\end{tabular} 

\subsection*{Problem 1.}

Find the solution of the following IVP using a substitution technique.
$$ \frac{\mathrm{d} y}{\mathrm{~d} x}=\frac{3 y-4 x}{x} \quad;\quad y(1)=8.$$

Try it out before looking at the solution.

\pagebreak

\subsection*{Solution to 1.}

Let's rewrite the equation and mess around with it to see if we spot the general form of an equation that we know:
$$ y' = \frac{3}{x}y - 4 \quad\rightarrow\quad y' - \frac{3}{x}y = -4. $$
It is FOL, but we want to use substitution. The reason we put it into FOL form is because this is how we can also identify a Bernoulli's equation, which has the general form:
$$ \underbrace{y' + p(x)y = f(x)y^{n}}_{\text{Bernoulli's Equation}}. $$
Our differential equation isn't of this form, so let's check for a different substitution technique. Namely, the homogenous substitution.
$$ \underbrace{y'(x, y) = y'(tx, ty)}_{\text{Homogenous Equation}}. $$
Let's perform this test with our differential equation:
$$ y' = \frac{3(ty) - 4(tx)}{(tx)} \quad\rightarrow\quad y' = \frac{3y-4x}{x} \left(\frac{t}{t}\right) \quad\therefore\quad y'(tx, ty)=y'(x,y). $$
Immediately, we should remember what our substitution factor is:
$$ \boxed{v=\frac{y}{x}} $$
Now, let's get our differential equation to consist of terms that contain a factor of $\frac{y}{x}$, so it's clear where we plug this substitution factor in.
$$ y' = 3\left(\frac{y}{x}\right) - 4. $$
Plugging in $v$ for $y/x$,
$$ y' = 3v -4. $$
This should look wrong. We need to find a way to replace the $y'$ with something in the $v$ world.
$$ v=\frac{y}{x} \quad\rightarrow\quad y = vx \quad\overset{\text{I.D. with product rule}}\rightarrow\quad y' = v'x + v. $$
Making this substitution for $y'$,
$$ v'x + v = 3v - 4 \quad\rightarrow\quad v'x = 2v-4 \quad\rightarrow\quad \frac{dv}{dx} = \frac{2v-4}{x}. $$
This is a separable equation, so let's separate it.
$$ \frac{1}{2v-4}dv = \frac{1}{x}dx \quad\overset{\int}\rightarrow\quad \frac{1}{2}\ln{|2v-4|} = \ln{|x|+C} \quad\overset{e}\rightarrow\quad 2v-4 = Ax^2 \quad\rightarrow\ldots$$
$$\ldots\rightarrow\quad v=\frac{1}{2}\left(Ax^2+4\right) \quad\rightarrow\quad v_g = Bx^2 + 2.  $$
This is the general solution for the differential equation, $v$, but we need a solution for $y$. Let's 'unsubstitute' our value of $v$.
$$ \frac{y_g}{x} = Bx^2 + 2 \quad\rightarrow\quad y_g = Bx^3 + 2x. $$
Let's use our initial condition: $y(1)=8.$
$$ (1,8)\quad:\quad 8 = B(1)^3 + 2(1) \quad\rightarrow\quad B = 6. $$
With this value of $B$,
$$ \underline{\boxed{y_s(x) = 6x^3 + 2x}} $$

\pagebreak

\subsection*{2.} 
Solve the IVP:
$$xy' + y = x^2y^2 \quad ; \quad y(1)=2.$$
\\
 
Try it out before looking at the solution.
\pagebreak

\subsubsection*{Solution to 2:}
Looking at this differential equation, we're not likely able to separate our variables, so let's try putting it into FOL form:
$$ xy' + y = x^2y^2 \quad\rightarrow \quad y' + \frac{y}{x} = xy^2$$
We'll pause here and think about what we have. Our $p(x)$ is $\frac{1}{x}$ and our $f(x)$ is some function of $x$ and $y$: $xy^2$. At this point we can rule our FOL since $f(x)$ is not a function of $x$ alone.\\

Since what we have is almost FOL, that tells us that we should try the Bernoulli's Equation form, which is:
$$ y' + p(x)y = f(x)y^n. $$
This is almost identical to FOL, but instead of $f(x)$ alone on the RHS of the equation, we have $f(x)y^n$.\\

For our equation we could turn $f(x,y)$ into $f(x)y^n$ where $n=2$ and $f(x)=x$. We now have
a Bernoulli's Equation. To begin with this solution technique, let's remember that our substitution should be:
$$ \boxed{v = y^{1-n}} $$
For our specific problem, $n=2$, so
$$ v = y^{-1}.$$
Now, let's take our substitution factor's implicit derivative to figure our what we should replace $y'$ with in our differential equation:
$$ v = y^{-1} \quad\overset{\frac{d}{dx}[v]}\rightarrow \quad v' = -y^{-2}y' \quad\rightarrow \quad y' = -y^2v'.$$

Let's make the underlined substitutions and see what our differential equation becomes:
$$ y' + \frac{1}{x}y = xy^2 \quad\rightarrow \quad [-y^2v'] + \frac{1}{x}y = xy^2. $$
It may seem like things have gotten worse, let's make our differential equation monic by dividing through by $-y^2$.
$$ \frac{-y^2v' + \frac{1}{x}y = xy^2}{-y^2} \quad\rightarrow \quad v' -\frac{1}{xy} = -x. $$
Lastly, since our substitution factor uses negative exponents, let's make what we have be in that same form (when working with Bernoulli equations, it's generally good to use negative exponents instead of fractions):
$$ v' - x^{-1}y^{-1} = -x \quad\text{if $v=y^{-1}$, then}\quad v' - x^{-1}v = -x. $$
Now, we have a FOL Equation where $p(x) = -\frac{1}{x}$ and $f(x)=-x$. Let's find our integrating factor.
$$ \mu(x) \equiv e^{\int p(x)dx} \quad\rightarrow \quad \mu(x) = e^{-\int \frac{1}{x} dx} \quad\rightarrow \quad \mu(x) = e^{\ln{(x^{-1})}} \quad\rightarrow \quad \mu(x) = x^{-1}. $$
Our solution form for FOL is:
$$ \mu v = \int \mu f.$$
For our problem, we have:
$$ x^{-1}v = \int x^{-1}(-x)dx \quad\rightarrow \quad v = x\int -1dx \quad\rightarrow \quad v = -x[x + C] \quad\rightarrow \ldots $$
$$\ldots\rightarrow \quad v = -x^2 + C_2x. $$
We're not done yet, this is a function of $v$, let's unsubstitute to make it a function of $y$:
$$ v(x) =  C_2x -x^2 \quad\rightarrow \quad y^{-1} = C_2x-x^2 \quad \rightarrow \quad y_g(x) = \frac{1}{C_2x-x^2}.$$
With our general solution, we can use the initial condition to find our specific solution:
$$ C: \quad y(1)=2\quad;\quad 2 = \frac{1}{C_2(1) - (1^2)} \quad\rightarrow \quad 2C_2 -2 = 1 \quad\rightarrow \quad  C_2 = \frac{3}{2}. $$
Plugging this into our general solution:
$$ \underline{\boxed{y_s(x) = \frac{1}{\frac{3}{2}x - x^2}}} $$




\end{document}